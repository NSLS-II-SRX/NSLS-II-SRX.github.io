%% Generated by Sphinx.
\def\sphinxdocclass{report}
\documentclass[letterpaper,10pt,english]{sphinxmanual}
\ifdefined\pdfpxdimen
   \let\sphinxpxdimen\pdfpxdimen\else\newdimen\sphinxpxdimen
\fi \sphinxpxdimen=.75bp\relax

\PassOptionsToPackage{warn}{textcomp}
\usepackage[utf8]{inputenc}
\ifdefined\DeclareUnicodeCharacter
% support both utf8 and utf8x syntaxes
\edef\sphinxdqmaybe{\ifdefined\DeclareUnicodeCharacterAsOptional\string"\fi}
  \DeclareUnicodeCharacter{\sphinxdqmaybe00A0}{\nobreakspace}
  \DeclareUnicodeCharacter{\sphinxdqmaybe2500}{\sphinxunichar{2500}}
  \DeclareUnicodeCharacter{\sphinxdqmaybe2502}{\sphinxunichar{2502}}
  \DeclareUnicodeCharacter{\sphinxdqmaybe2514}{\sphinxunichar{2514}}
  \DeclareUnicodeCharacter{\sphinxdqmaybe251C}{\sphinxunichar{251C}}
  \DeclareUnicodeCharacter{\sphinxdqmaybe2572}{\textbackslash}
\fi
\usepackage{cmap}
\usepackage[T1]{fontenc}
\usepackage{amsmath,amssymb,amstext}
\usepackage{babel}
\usepackage{times}
\usepackage[Bjarne]{fncychap}
\usepackage{sphinx}

\fvset{fontsize=\small}
\usepackage{geometry}

% Include hyperref last.
\usepackage{hyperref}
% Fix anchor placement for figures with captions.
\usepackage{hypcap}% it must be loaded after hyperref.
% Set up styles of URL: it should be placed after hyperref.
\urlstyle{same}
\addto\captionsenglish{\renewcommand{\contentsname}{Contents:}}

\addto\captionsenglish{\renewcommand{\figurename}{Fig.\@ }}
\makeatletter
\def\fnum@figure{\figurename\thefigure{}}
\makeatother
\addto\captionsenglish{\renewcommand{\tablename}{Table }}
\makeatletter
\def\fnum@table{\tablename\thetable{}}
\makeatother
\addto\captionsenglish{\renewcommand{\literalblockname}{Listing}}

\addto\captionsenglish{\renewcommand{\literalblockcontinuedname}{continued from previous page}}
\addto\captionsenglish{\renewcommand{\literalblockcontinuesname}{continues on next page}}
\addto\captionsenglish{\renewcommand{\sphinxnonalphabeticalgroupname}{Non-alphabetical}}
\addto\captionsenglish{\renewcommand{\sphinxsymbolsname}{Symbols}}
\addto\captionsenglish{\renewcommand{\sphinxnumbersname}{Numbers}}

\addto\extrasenglish{\def\pageautorefname{page}}

\setcounter{tocdepth}{0}



\title{NSLS-II SRX Documentation}
\date{Apr 06, 2021}
\release{}
\author{SRX}
\newcommand{\sphinxlogo}{\vbox{}}
\renewcommand{\releasename}{}
\makeindex
\begin{document}

\pagestyle{empty}
\sphinxmaketitle
\pagestyle{plain}
\sphinxtableofcontents
\pagestyle{normal}
\phantomsection\label{\detokenize{index::doc}}



\chapter{Introduction to SRX}
\label{\detokenize{intro:introduction-to-srx}}\label{\detokenize{intro::doc}}
The Sub-micron Resolution X-ray Spectroscopy (SRX) beamline at the NSLS-II supports a wide variety of scientific use-cases, ranging from geoscience through energy materials. Facilities include state-of-the-art, sub-micron focusing X-ray optics and a flexible sample environment.

\begin{figure}[htbp]
\centering
\capstart

\sphinxhref{\_images/SRX\_BeamlineSchematic\_text\_tab\_web.png}{\sphinxincludegraphics[width=1.000\linewidth]{{SRX_BeamlineSchematic_text_tab_web}.png}}
\caption{A schematic of the SRX beamline at the NSLS-II.}\label{\detokenize{intro:id1}}\label{\detokenize{intro:fig-prompt}}\end{figure}


\chapter{Useful Commands}
\label{\detokenize{useful:useful-commands}}\label{\detokenize{useful::doc}}
Below is a list of useful commands for running the SRX beamline. Previous commands can be seen by hitting the up arrow in Bluesky. To search through them, you can start typing a command before hitting the up arrow to filter your history.


\section{Starting Bluesky}
\label{\detokenize{useful:starting-bluesky}}
Start Bluesky - \sphinxstyleemphasis{Start Bluesky from the terminal.}

\begin{sphinxVerbatim}[commandchars=\\\{\}]
\PYGZdl{} bsui
\end{sphinxVerbatim}


\section{General Functions}
\label{\detokenize{useful:general-functions}}
Change X-ray energy - \sphinxstyleemphasis{Either command can be used below. The energy can be entered in units of eV or keV.}

\begin{sphinxVerbatim}[commandchars=\\\{\}]
\PYG{n}{Bluesky}\PYG{n+nd}{@SRX} \PYG{p}{[}\PYG{l+m+mi}{1}\PYG{p}{]} \PYG{n}{energy}\PYG{o}{.}\PYG{n}{move}\PYG{p}{(}\PYG{l+m+mf}{7.2}\PYG{p}{)}
\PYG{n}{Bluesky}\PYG{n+nd}{@SRX} \PYG{p}{[}\PYG{l+m+mi}{2}\PYG{p}{]} \PYG{o}{\PYGZpc{}}\PYG{n}{mov} \PYG{n}{energy} \PYG{l+m+mf}{7.2}
\end{sphinxVerbatim}

Optimize the beam - \sphinxstyleemphasis{Maximize the X-ray flux.}

\begin{sphinxVerbatim}[commandchars=\\\{\}]
\PYG{n}{RE}\PYG{p}{(}\PYG{n}{peakup\PYGZus{}fine}\PYG{p}{(}\PYG{p}{)}\PYG{p}{)}
\end{sphinxVerbatim}

Setting a region of interest - \sphinxstyleemphasis{Set the ROI on the detector. The specific edge is optional.}

\begin{sphinxVerbatim}[commandchars=\\\{\}]
\PYG{n}{Bluesky}\PYG{n+nd}{@SRX} \PYG{p}{[}\PYG{l+m+mi}{1}\PYG{p}{]} \PYG{n}{setroi}\PYG{p}{(}\PYG{l+m+mi}{1}\PYG{p}{,} \PYG{l+s+s1}{\PYGZsq{}}\PYG{l+s+s1}{Fe}\PYG{l+s+s1}{\PYGZsq{}}\PYG{p}{)}
\PYG{n}{Bluesky}\PYG{n+nd}{@SRX} \PYG{p}{[}\PYG{l+m+mi}{2}\PYG{p}{]} \PYG{n}{setroi}\PYG{p}{(}\PYG{l+m+mi}{1}\PYG{p}{,} \PYG{l+s+s1}{\PYGZsq{}}\PYG{l+s+s1}{Fe}\PYG{l+s+s1}{\PYGZsq{}}\PYG{p}{,} \PYG{l+s+s1}{\PYGZsq{}}\PYG{l+s+s1}{ka1}\PYG{l+s+s1}{\PYGZsq{}}\PYG{p}{)}
\end{sphinxVerbatim}


\section{XRF Imaging}
\label{\detokenize{useful:xrf-imaging}}
Fly scan - \sphinxstyleemphasis{Perform a fly scan. Return an image with dimensions (numX, numY)}

\begin{sphinxVerbatim}[commandchars=\\\{\}]
\PYG{n}{Bluesky}\PYG{n+nd}{@SRX} \PYG{p}{[}\PYG{l+m+mi}{1}\PYG{p}{]} \PYG{n}{RE}\PYG{p}{(}\PYG{n}{nano\PYGZus{}scan\PYGZus{}and\PYGZus{}fly}\PYG{p}{(}\PYG{n}{startX}\PYG{p}{,} \PYG{n}{stopX}\PYG{p}{,} \PYG{n}{numX}\PYG{p}{,}
                                     \PYG{n}{startY}\PYG{p}{,} \PYG{n}{stopY}\PYG{p}{,} \PYG{n}{numY}\PYG{p}{,} \PYG{n}{dwell}\PYG{p}{)}\PYG{p}{)}
\PYG{n}{Bluesky}\PYG{n+nd}{@SRX} \PYG{p}{[}\PYG{l+m+mi}{2}\PYG{p}{]} \PYG{n}{RE}\PYG{p}{(}\PYG{n}{nano\PYGZus{}y\PYGZus{}scan\PYGZus{}and\PYGZus{}fly}\PYG{p}{(}\PYG{n}{startY}\PYG{p}{,} \PYG{n}{stopY}\PYG{p}{,} \PYG{n}{numY}\PYG{p}{,}
                                       \PYG{n}{startX}\PYG{p}{,} \PYG{n}{stopX}\PYG{p}{,} \PYG{n}{numX}\PYG{p}{,} \PYG{n}{dwell}\PYG{p}{)}\PYG{p}{)}
\end{sphinxVerbatim}

Step scan - \sphinxstyleemphasis{Perform a step scan. Note: these arguments take a step size, not the number of points.}

\begin{sphinxVerbatim}[commandchars=\\\{\}]
\PYG{n}{Bluesky}\PYG{n+nd}{@SRX} \PYG{p}{[}\PYG{l+m+mi}{1}\PYG{p}{]} \PYG{n}{RE}\PYG{p}{(}\PYG{n}{nano\PYGZus{}xrf}\PYG{p}{(}\PYG{n}{startX}\PYG{p}{,} \PYG{n}{stopX}\PYG{p}{,} \PYG{n}{stepX}\PYG{p}{,}
                            \PYG{n}{startY}\PYG{p}{,} \PYG{n}{stopY}\PYG{p}{,} \PYG{n}{stepY}\PYG{p}{,} \PYG{n}{dwell}\PYG{p}{)}\PYG{p}{)}
\end{sphinxVerbatim}


\section{XAS Spectroscopy}
\label{\detokenize{useful:xas-spectroscopy}}
Print element binding energies - \sphinxstyleemphasis{Print the binding energies for the element of interest. The “best” edge can be returned as available.}

\begin{sphinxVerbatim}[commandchars=\\\{\}]
\PYG{n}{Bluesky}\PYG{n+nd}{@SRX} \PYG{p}{[}\PYG{l+m+mi}{1}\PYG{p}{]} \PYG{n}{Fe\PYGZus{}k} \PYG{o}{=} \PYG{n}{getbindingE}\PYG{p}{(}\PYG{l+s+s1}{\PYGZsq{}}\PYG{l+s+s1}{Fe}\PYG{l+s+s1}{\PYGZsq{}}\PYG{p}{)}
\end{sphinxVerbatim}

Print element emission energies - \sphinxstyleemphasis{Print the emission energies for the element of interest.}

\begin{sphinxVerbatim}[commandchars=\\\{\}]
\PYG{n}{Bluesky}\PYG{n+nd}{@SRX} \PYG{p}{[}\PYG{l+m+mi}{1}\PYG{p}{]} \PYG{n}{getemissionE}\PYG{p}{(}\PYG{l+s+s1}{\PYGZsq{}}\PYG{l+s+s1}{Fe}\PYG{l+s+s1}{\PYGZsq{}}\PYG{p}{)}
\end{sphinxVerbatim}

XANES scan - \sphinxstyleemphasis{Run a XANES scan. This scan has 3 regions with different steps spanning the iron K-edge.}

\begin{sphinxVerbatim}[commandchars=\\\{\}]
\PYG{n}{Bluesky}\PYG{n+nd}{@SRX} \PYG{p}{[}\PYG{l+m+mi}{1}\PYG{p}{]} \PYG{n}{RE}\PYG{p}{(}\PYG{n}{xanes\PYGZus{}plan}\PYG{p}{(}\PYG{n}{erange}\PYG{o}{=}\PYG{p}{[}\PYG{n}{Fe\PYGZus{}k}\PYG{o}{\PYGZhy{}}\PYG{l+m+mi}{50}\PYG{p}{,} \PYG{n}{Fe\PYGZus{}k}\PYG{o}{\PYGZhy{}}\PYG{l+m+mi}{10}\PYG{p}{,} \PYG{n}{Fe\PYGZus{}k}\PYG{o}{+}\PYG{l+m+mi}{50}\PYG{p}{,} \PYG{n}{Fe\PYGZus{}k}\PYG{o}{+}\PYG{l+m+mi}{150}\PYG{p}{]}\PYG{p}{,}
                              \PYG{n}{estep}\PYG{o}{=}\PYG{p}{[}\PYG{l+m+mf}{2.0}\PYG{p}{,} \PYG{l+m+mf}{1.0}\PYG{p}{,} \PYG{l+m+mf}{2.0}\PYG{p}{]}\PYG{p}{,}
                              \PYG{n}{acqtime}\PYG{o}{=}\PYG{l+m+mf}{1.0}\PYG{p}{,}
                              \PYG{n}{samplename}\PYG{o}{=}\PYG{l+s+s1}{\PYGZsq{}}\PYG{l+s+s1}{Fe foil}\PYG{l+s+s1}{\PYGZsq{}}\PYG{p}{,}
                              \PYG{n}{filename}\PYG{o}{=}\PYG{l+s+s1}{\PYGZsq{}}\PYG{l+s+s1}{Fe\PYGZus{}foil}\PYG{l+s+s1}{\PYGZsq{}}\PYG{p}{]}\PYG{p}{)}\PYG{p}{)}
\end{sphinxVerbatim}


\section{Troubleshooting}
\label{\detokenize{useful:troubleshooting}}
Pause a scan - \sphinxstyleemphasis{The scan will pause at the next checkpoint.}

\begin{sphinxVerbatim}[commandchars=\\\{\}]
\PYG{n}{CTRL}\PYG{o}{+}\PYG{n}{C}
\end{sphinxVerbatim}

Urgently stop a scan - \sphinxstyleemphasis{With each CTRL-C, Bluesky raisens the urgency of stopping the scan.}

\begin{sphinxVerbatim}[commandchars=\\\{\}]
\PYG{n}{CTRL}\PYG{o}{+}\PYG{n}{C} \PYG{n}{x20}
\end{sphinxVerbatim}

Resume a scan - \sphinxstyleemphasis{A scan can be resumed after pausing.}

\begin{sphinxVerbatim}[commandchars=\\\{\}]
\PYG{n}{Bluesky}\PYG{n+nd}{@SRX} \PYG{p}{[}\PYG{l+m+mi}{1}\PYG{p}{]} \PYG{n}{RE}\PYG{o}{.}\PYG{n}{resume}\PYG{p}{(}\PYG{p}{)}
\end{sphinxVerbatim}

Stop a scan - \sphinxstyleemphasis{Stop a scan and label the scan as a success or failure.}

\begin{sphinxVerbatim}[commandchars=\\\{\}]
\PYG{n}{Bluesky}\PYG{n+nd}{@SRX} \PYG{p}{[}\PYG{l+m+mi}{1}\PYG{p}{]} \PYG{n}{RE}\PYG{o}{.}\PYG{n}{stop}\PYG{p}{(}\PYG{p}{)}   \PYG{c+c1}{\PYGZsh{} Label scan as success}
\PYG{n}{Bluesky}\PYG{n+nd}{@SRX} \PYG{p}{[}\PYG{l+m+mi}{2}\PYG{p}{]} \PYG{n}{RE}\PYG{o}{.}\PYG{n}{abort}\PYG{p}{(}\PYG{p}{)}  \PYG{c+c1}{\PYGZsh{} Label scan as failure}
\end{sphinxVerbatim}


\chapter{Beamline Staff Pages {[}staff only{]}}
\label{\detokenize{staff:beamline-staff-pages-staff-only}}\label{\detokenize{staff::doc}}
The information provided on this page is directed towards helping beamline staff.


\section{Quick Links}
\label{\detokenize{staff:quick-links}}\begin{enumerate}
\def\theenumi{\arabic{enumi}}
\def\labelenumi{\theenumi .}
\makeatletter\def\p@enumii{\p@enumi \theenumi .}\makeatother
\item {} 
\sphinxhref{http://xf05idd-webcam1.nsls2.bnl.local}{Analog camera web server}

\item {} 
\sphinxhref{http://xf05idd-webcam2.nsls2.bnl.local}{Ceiling camera web server}

\item {} 
\sphinxhref{https://controlsweb.nsls2.bnl.gov/trac/}{Trac ticket system}

\item {} 
\sphinxhref{http://jira.nsls2.bnl.gov}{Jira ticket system}

\item {} 
\sphinxhref{https://nsls2bid.bnl.gov/SAF/Index/5-ID}{SRX Posted SAFs}

\item {} 
\sphinxhref{http://xf05id2-ca1.nsls2.bnl.local/logbook/index.html}{SRX Olog}

\item {} 
\sphinxhref{https://ps.bnl.gov/docs/Reference/NSLS-II\%20Beamline\%205-ID\%20Radiation\%20Safety\%20Component\%20Checklist\%20TEMPLATE.pdf}{SRX Radiation Safety Component Checklist}

\item {} 
\sphinxhref{http://nsls-ii.github.io/}{NSLS-II Controls Documentation}

\item {} 
\sphinxhref{https://controlsweb01.nsls2.bnl.gov/IP/?page=login\&section=timeout}{NSLS-II IP Address Management}

\item {} 
\sphinxhref{https://ps.bnl.gov/phot/ros/Shared\%20Documents/MAXIMO\%20Development/Beamlines/LT-R-XFD-CO-DR-SRX-002\_Rev1.xlsx}{NSLS-II Sharepoint Documentation}

\item {} 
\sphinxhref{https://beamline5id.bnl.gov/index.php/Main\_Page}{Former SRX Wiki Staff Page}

\end{enumerate}

\begin{sphinxadmonition}{note}{\label{staff:index-0}Todo:}\begin{itemize}
\item {} 
Fix IOCs reference

\end{itemize}
\end{sphinxadmonition}


\section{Setting up Users}
\label{\detokenize{staff:setting-up-users}}\begin{enumerate}
\def\theenumi{\arabic{enumi}}
\def\labelenumi{\theenumi .}
\makeatletter\def\p@enumii{\p@enumi \theenumi .}\makeatother
\item {} 
Post SAF to \sphinxhref{http://passadmin.bnl.gov}{PASS}

\item {} \begin{description}
\item[{Update user-specific metadata}] \leavevmode\begin{itemize}
\item {} 
Open /nsls2/users/xf05id1/.ipython/profile\_collection/startup/90-userdata.py

\item {} 
Update proposal dictionary with information from posted SAF. Save.

\item {} 
Restart bluesky.

\end{itemize}

\end{description}

\item {} 
Perform beamline specific training. \sphinxhref{https://www.bnl.gov/ps/training/Beamline-BST-Forms/PS-BST-5-ID.pdf}{5-ID BST Form}

\end{enumerate}


\section{Beamline Setup}
\label{\detokenize{staff:beamline-setup}}
\sphinxstyleemphasis{These tasks are typically done once a cycle}

\begin{sphinxadmonition}{note}{\label{staff:index-1}Todo:}\begin{itemize}
\item {} 
Beamline alignment

\item {} 
Setting up the Merlin

\item {} 
Setting up the Dexela

\item {} 
Setting up the Xspress3

\end{itemize}
\end{sphinxadmonition}


\subsection{Preparing for a new cycle}
\label{\detokenize{staff:preparing-for-a-new-cycle}}\begin{description}
\item[{\sphinxstyleemphasis{This is a comprehensive list of things to consider before the start of a cycle.}}] \leavevmode\begin{itemize}
\item {} 
Close all system safety work permits.

\item {} 
Check cryocooler pressure and load.

\item {} 
Check all vacuum, temperature, water systems.

\item {} 
Check and top-off the PPS burn-through.

\item {} 
Confirm the RGA in the A-hutch is connected and scanning.

\item {} 
Perform a Radiation Safety Component Checklist.

\item {} 
Post a valid SAF and ESR.

\item {} 
Test and deploy the latest bluesky environment.

\item {} 
Setup lsyncd.

\end{itemize}

\end{description}


\subsection{Aligning the Beamline}
\label{\detokenize{staff:aligning-the-beamline}}\begin{description}
\item[{\sphinxstyleemphasis{Historically, the beamline and storage ring take about a day to stabilize. Therefore, on day 1 of operations, it makes sense to open the front-end shutter and get light through the monochromator. Since components will drift, optimization should take place on day 2 and after a local bump is performed.}}] \leavevmode\begin{enumerate}
\def\theenumi{\arabic{enumi}}
\def\labelenumi{\theenumi .}
\makeatletter\def\p@enumii{\p@enumi \theenumi .}\makeatother
\item {} 
Previous motor positions should be captured at the end of each cycle. As a precaution, capture the current motor positions.

\item {} \begin{description}
\item[{Check front-end (FE) slits, white-beam (WB) slits, and mirror (HFM) position.}] \leavevmode\begin{itemize}
\item {} 
Open the gap of the undulator to 18 000 \(\mu\)m. Insert the camera in the HFM tank.

\item {} 
Open the FE shutter. Open the WB slits all the way (4 mm x 4 mm). Turn off mirror pitch feedback and reset the voltage to 30 V. Remove the HFM by dropping the pitch to 0.0 mrad and translating in the positive direction by 3 mm.

\item {} 
Tweak the FE slits so the slits are just protecting the mask.

\item {} 
Tweak the WB slits so the slits are centered. Return the WB slit gaps to their previous values (0.5 mm V x 2.0 mm H).

\item {} 
Bring the HFM back in. Center the mirror on the incoming beam and confirm the mirror is parallel to the beam. Pitch to the nominal 2.5 mrad. Enable the mirror pitch feedback.

\item {} 
Close the FE shutter and retract the camera.

\end{itemize}

\end{description}

\item {} \begin{description}
\item[{Align the monochromator to allow light through.}] \leavevmode\begin{itemize}
\item {} 
With the FE shutter closed, insert the BPM 1 camera (\sphinxstyleemphasis{this can be slow}).

\item {} 
In bluesky, set the energy to the last used value. If starting from scratch, choose a higher energy such as 12 keV. By using bluesky, this will set the undulator gap, and monochomator positions to a reasonably close value.

\item {} 
Open the pink-beam (PB) slits to a 4.0 mm gap to make sure they are fully open.

\item {} 
Open the FE shutter and hopefully light will come through onto the camera.

\item {} 
If not, set the exposure time on the camera to something large, like 0.1 s. This will help you see the light come through while you scan the motors. There are 4 motors that can be off: Bragg, crystal offset, roll, and pitch. Hopefully, by starting with Bragg you can start to see some light and then optimize by tweaking pitch and roll. Finally, the position of the beam can be translated with the crystal offset.

\item {} 
The PB slits can be centered and closed so they are just intercepting the beam.

\item {} 
Once the light is through the monochromator, the FE shutter can be closed, the camera removed, and alignment downstream can continue.

\end{itemize}

\end{description}

\item {} 
Tweak monochromator and mirror alignment to center secondary source aperture (SSA).

\end{enumerate}

\end{description}


\subsection{Calibrating the monochromator}
\label{\detokenize{staff:calibrating-the-monochromator}}\begin{description}
\item[{\sphinxstyleemphasis{Calibrating the monochromator is done by collecting XANES spectra across several element absorption edges. A least-squares fitting routine will then calculate the HDCM parameters for the calibration}}] \leavevmode\begin{enumerate}
\def\theenumi{\arabic{enumi}}
\def\labelenumi{\theenumi .}
\makeatletter\def\p@enumii{\p@enumi \theenumi .}\makeatother
\item {} 
Collect XANES scans at 3-5 different energies. For the best fit, a wide range of energies is best. Typically, scans are performed using V, Cr, Fe, Cu, Se, Zr foils. \sphinxstyleemphasis{It is a good idea to record the C1 Roll and C2 Pitch values for each energy. These can be used for a lookup table to improve the peakup function.}:

\begin{sphinxVerbatim}[commandchars=\\\{\}]
\PYG{n}{Bluesky}\PYG{n+nd}{@SRX} \PYG{p}{[}\PYG{l+m+mi}{1}\PYG{p}{]} \PYG{n}{X} \PYG{o}{=} \PYG{n}{getbindingE}\PYG{p}{(}\PYG{l+s+s1}{\PYGZsq{}}\PYG{l+s+s1}{Fe}\PYG{l+s+s1}{\PYGZsq{}}\PYG{p}{)}
\PYG{n}{Bluesky}\PYG{n+nd}{@SRX} \PYG{p}{[}\PYG{l+m+mi}{2}\PYG{p}{]} \PYG{o}{\PYGZpc{}}\PYG{n}{mov} \PYG{n}{energy} \PYG{n}{X}
\PYG{n}{Bluesky}\PYG{n+nd}{@SRX} \PYG{p}{[}\PYG{l+m+mi}{3}\PYG{p}{]} \PYG{n}{RE}\PYG{p}{(}\PYG{n}{peakup\PYGZus{}fine}\PYG{p}{(}\PYG{p}{)}\PYG{p}{)}
\PYG{n}{Bluesky}\PYG{n+nd}{@SRX} \PYG{p}{[}\PYG{l+m+mi}{4}\PYG{p}{]} \PYG{n}{RE}\PYG{p}{(}\PYG{n}{xanes\PYGZus{}plan}\PYG{p}{(}\PYG{p}{[}\PYG{n}{X}\PYG{o}{\PYGZhy{}}\PYG{l+m+mi}{50}\PYG{p}{,} \PYG{n}{X}\PYG{o}{+}\PYG{l+m+mi}{50}\PYG{p}{]}\PYG{p}{,} \PYG{p}{[}\PYG{l+m+mi}{1}\PYG{p}{]}\PYG{p}{,} \PYG{l+m+mf}{0.1}\PYG{p}{)}\PYG{p}{)}
\end{sphinxVerbatim}

\item {} 
Define a dictionary in bluesky with element symbols mapped to scan IDs.:

\begin{sphinxVerbatim}[commandchars=\\\{\}]
\PYG{n}{Bluesky}\PYG{n+nd}{@SRX} \PYG{p}{[}\PYG{l+m+mi}{5}\PYG{p}{]} \PYG{n}{scanlogDic} \PYG{o}{=} \PYG{p}{\PYGZob{}}\PYG{l+s+s1}{\PYGZsq{}}\PYG{l+s+s1}{V}\PYG{l+s+s1}{\PYGZsq{}} \PYG{p}{:} \PYG{l+m+mi}{1000}\PYG{p}{,}
                              \PYG{l+s+s1}{\PYGZsq{}}\PYG{l+s+s1}{Cr}\PYG{l+s+s1}{\PYGZsq{}}\PYG{p}{:} \PYG{l+m+mi}{1001}\PYG{p}{,}
                              \PYG{l+s+s1}{\PYGZsq{}}\PYG{l+s+s1}{Fe}\PYG{l+s+s1}{\PYGZsq{}}\PYG{p}{:} \PYG{l+m+mi}{1002}\PYG{p}{,}
                              \PYG{l+s+s1}{\PYGZsq{}}\PYG{l+s+s1}{Cu}\PYG{l+s+s1}{\PYGZsq{}}\PYG{p}{:} \PYG{l+m+mi}{1003}\PYG{p}{,}
                              \PYG{l+s+s1}{\PYGZsq{}}\PYG{l+s+s1}{Se}\PYG{l+s+s1}{\PYGZsq{}}\PYG{p}{:} \PYG{l+m+mi}{1004}\PYG{p}{,}
                              \PYG{l+s+s1}{\PYGZsq{}}\PYG{l+s+s1}{Zr}\PYG{l+s+s1}{\PYGZsq{}}\PYG{p}{:} \PYG{l+m+mi}{1005}\PYG{p}{\PYGZcb{}}
\end{sphinxVerbatim}

\item {} 
Run the \sphinxstyleemphasis{braggcalib()} function with the dictionary as input. The function will go through each scan and display a plot marking where the edge was found. Finally, this will output the new HDCM parameters.:

\begin{sphinxVerbatim}[commandchars=\\\{\}]
\PYG{n}{Bluesky}\PYG{n+nd}{@SRX} \PYG{p}{[}\PYG{l+m+mi}{6}\PYG{p}{]} \PYG{n}{braggcalib}\PYG{p}{(}\PYG{n}{scanlogDic}\PYG{o}{=}\PYG{n}{scanlogDic}\PYG{p}{,} \PYG{n}{use\PYGZus{}xrf}\PYG{o}{=}\PYG{k+kc}{True}\PYG{p}{)}
\end{sphinxVerbatim}

\item {} 
Update the values in the bluesky profile (10-machine.py). Save and restart bluesky.

\end{enumerate}

\end{description}


\section{Beamline Maintenance}
\label{\detokenize{staff:beamline-maintenance}}
\begin{sphinxadmonition}{note}{\label{staff:index-2}Todo:}\begin{itemize}
\item {} 
Calibrating the Xspress3

\item {} 
Power loss preparation and recovery

\end{itemize}
\end{sphinxadmonition}


\subsection{Cryocooler}
\label{\detokenize{staff:cryocooler}}
\sphinxstyleemphasis{The manual for the cryocooler can be found here.}

\begin{sphinxadmonition}{note}{\label{staff:index-3}Todo:}\begin{itemize}
\item {} 
Upload cryocooler manual

\end{itemize}
\end{sphinxadmonition}


\subsubsection{Warming the cryocooler}
\label{\detokenize{staff:warming-the-cryocooler}}\begin{enumerate}
\def\theenumi{\arabic{enumi}}
\def\labelenumi{\theenumi .}
\makeatletter\def\p@enumii{\p@enumi \theenumi .}\makeatother
\item {} 
Connect a turbo-pump station to the monochromator tank. Pump the turbo so that it reads a pressure in the 10$^{\text{-8}}$ Torr range.

\item {} 
Close the beamline gate valves to isolate the monochromator.

\item {} 
Open the manual valve between the monochromator and turbo-pump.

\item {} 
Turn off the ion pump. This should automatically put the cryocooler in “Stop” mode.

\item {} 
The cryocooler will warm up over several days.

\end{enumerate}


\subsubsection{Cooling the cryocooler}
\label{\detokenize{staff:cooling-the-cryocooler}}\begin{enumerate}
\def\theenumi{\arabic{enumi}}
\def\labelenumi{\theenumi .}
\makeatletter\def\p@enumii{\p@enumi \theenumi .}\makeatother
\item {} 
Verify the monochromator cold cathode gauge is on and the cryocooler is not inhibitted. This typically means a pressure reading better than 10$^{\text{-7}}$ Torr.

\item {} \begin{description}
\item[{Purge the system according to the cryocooler manual, section 3.3.1, on page 28. \sphinxstyleemphasis{Note: V10 and V11 are variable values, 0\% = Close, 100\% = Open.}}] \leavevmode\begin{itemize}
\item {} 
Verify N$_{\text{2}}$ gas source is at a pressure between 1.5 and 3.0 bar.

\item {} 
Verify the monochromator ion pump and cold cathode gauge are on.

\item {} 
Close all the valves.

\item {} 
Open V9, V10, V20, and V21 fully. Purge for 30 min.

\item {} 
Close V9 and open V11. Purge for 15 min.

\item {} 
Close V11. Open V17 and purge for 15 min.

\item {} 
Close all the valves.

\end{itemize}

\end{description}

\item {} \begin{description}
\item[{Following the cryocooler manual, section 4.2.1.1, fill the sub-cooler to 15\% and fill the heater vessel to 20\%.}] \leavevmode\begin{itemize}
\item {} 
Verify the liquid N$_{\text{2}}$ source valve is open.

\item {} 
Open V19 to start filling the sub-cooler.

\item {} 
Close V19 when the sub-cooler reaches 15\%.

\item {} 
Open heater vessel valve to start filling the heater vessel.

\item {} 
Close heater vessel valve when level reaches 20\%.

\end{itemize}

\end{description}

\item {} \begin{description}
\item[{Follow the automatic cool down proceedure from the manual, section 4.3.}] \leavevmode\begin{itemize}
\item {} 
Verify the liquid N$_{\text{2}}$ source valve is open.

\item {} 
From the cryocooler CSS page, click “Cool Down”.

\item {} 
Once full, in CSS click on the “A” to enable automatic filling of the cryocooler.

\end{itemize}

\end{description}

\end{enumerate}


\section{Controls}
\label{\detokenize{staff:controls}}

\subsection{IOC Monitoring}
\label{\detokenize{staff:ioc-monitoring}}\begin{description}
\item[{On a Debian server, the manage-iocs tool can be used to monitor the IOC status. SSH into the server that hosts the IOC (\sphinxstyleemphasis{e.g.} xf05idd-ioc1) and run:}] \leavevmode\begin{itemize}
\item {} 
List all IOCs

\begin{sphinxVerbatim}[commandchars=\\\{\}]
\PYGZdl{} manage\PYGZhy{}iocs report
\end{sphinxVerbatim}

\item {} 
Show IOC status

\begin{sphinxVerbatim}[commandchars=\\\{\}]
\PYGZdl{} manage\PYGZhy{}iocs status
\end{sphinxVerbatim}

\item {} 
Start IOC, \sphinxstyleemphasis{softioc-example}. The path to the IOC can be found using \sphinxcode{\sphinxupquote{manage-iocs}}.

\begin{sphinxVerbatim}[commandchars=\\\{\}]
\PYGZdl{} sudo /etc/init.d/softioc\PYGZhy{}example start
\end{sphinxVerbatim}

\item {} 
Stop IOC, \sphinxstyleemphasis{softioc-example}. The path to the IOC can be found using \sphinxcode{\sphinxupquote{manage-iocs}}.

\begin{sphinxVerbatim}[commandchars=\\\{\}]
\PYGZdl{} sudo /etc/init.d/softioc\PYGZhy{}example stop
\end{sphinxVerbatim}

\end{itemize}

\item[{On a CentOS server, the IOCs are managed using procServ. This is typically a simple executable script that will start them.}] \leavevmode\begin{itemize}
\item {} 
SSH into the camera server, xf05idd-ioc2.

\item {} 
To start the IOC for the Blackfly camera

\begin{sphinxVerbatim}[commandchars=\\\{\}]
\PYGZdl{} cd /epics/iocs/cam\PYGZhy{}bfly1
\PYGZdl{} ./start\PYGZus{}cam\PYGZus{}bfly1
\end{sphinxVerbatim}

\item {} 
Using these commands, the IOC will start and you will be in a telnet of the IOC.

\item {} 
To exit the telnet, type \sphinxcode{\sphinxupquote{Ctrl+{]}}} and then \sphinxcode{\sphinxupquote{q}}.

\item {} 
To stop the IOC, the process for procServ must be stopped. The process ID is the second column.

\begin{sphinxVerbatim}[commandchars=\\\{\}]
\PYGZdl{} ps aux \textbar{} grep procServ
akiss      820  0.0  0.0  27448   976 ?        Ss    2020  29:07 procServ \PYGZhy{}\PYGZhy{}logstamp \PYGZhy{}n cam\PYGZhy{}bfly1 \PYGZhy{}i \PYGZca{}D \PYGZhy{}L /epics/iocs/cam\PYGZhy{}bfly1/log/cam\PYGZhy{}bfly1.log 20001 ./st.cmd
\PYGZdl{} sudo kill \PYGZhy{}9 820
\end{sphinxVerbatim}

\end{itemize}

\end{description}


\subsection{Motion Controls}
\label{\detokenize{staff:motion-controls}}
\begin{sphinxadmonition}{note}{\label{staff:index-4}Todo:}\begin{itemize}
\item {} 
Insert table with: Motor controller, IOC, Motor, PV, Bluesky object

\item {} 
Rearrange table to be motor, bluesky, IOC, controller, PV?

\item {} 
List of all IOCs on each server

\end{itemize}
\end{sphinxadmonition}


\begin{savenotes}\sphinxattablestart
\raggedright
\sphinxcapstartof{table}
\sphinxthecaptionisattop
\sphinxcaption{xf05ida-ioc1 motors}\label{\detokenize{staff:xf05ida-ioc1-motors}}
\sphinxaftertopcaption
\begin{tabulary}{\linewidth}[t]{|T|T|T|T|T|}
\hline
\sphinxstyletheadfamily 
Motor Controller
&\sphinxstyletheadfamily 
IOC
&\sphinxstyletheadfamily 
Motor
&\sphinxstyletheadfamily 
PV
&\sphinxstyletheadfamily 
Bluesky Object
\\
\hline
mc01
&
softioc-mc01
&
testmotor
&
XF:
&
bs.motor
\\
\hline
\end{tabulary}
\par
\sphinxattableend\end{savenotes}


\begin{savenotes}\sphinxattablestart
\raggedright
\sphinxcapstartof{table}
\sphinxthecaptionisattop
\sphinxcaption{xf05idd-ioc1 motors}\label{\detokenize{staff:xf05idd-ioc1-motors}}
\sphinxaftertopcaption
\begin{tabulary}{\linewidth}[t]{|T|T|T|T|T|}
\hline
\sphinxstyletheadfamily 
Motor Controller
&\sphinxstyletheadfamily 
IOC
&\sphinxstyletheadfamily 
Motor
&\sphinxstyletheadfamily 
PV
&\sphinxstyletheadfamily 
Bluesky Object
\\
\hline
mc01
&
softioc-mc01
&
testmotor
&
XF:
&
bs.motor
\\
\hline
\end{tabulary}
\par
\sphinxattableend\end{savenotes}


\begin{savenotes}\sphinxattablestart
\raggedright
\sphinxcapstartof{table}
\sphinxthecaptionisattop
\sphinxcaption{xf05idd-ioc3 motors}\label{\detokenize{staff:xf05idd-ioc3-motors}}
\sphinxaftertopcaption
\begin{tabulary}{\linewidth}[t]{|T|T|T|T|T|}
\hline
\sphinxstyletheadfamily 
Motor
&\sphinxstyletheadfamily 
Bluesky Object
&\sphinxstyletheadfamily 
Motor Controller
&\sphinxstyletheadfamily 
IOC
&\sphinxstyletheadfamily 
PV
\\
\hline
nanoKBv angle calc
&
bs.motor
&
none
&
softioc-anglecalc
&
XF:05IDD-ES:1\{nKB:vert-Ax:PC\}Mtr
\\
\hline
nanoKBh angle calc
&
bs.motor
&
none
&
softioc-anglecalc
&
XF:05IDD-ES:1\{nKB:horz-Ax:PC\}Mtr
\\
\hline
testmotor
&
bs.motor
&
fpsensor1
&
softioc-fpsensor
&
XF:05IDD-ES:1\{FPS:1-Chan0\}Pos-I
\\
\hline
testmotor
&
bs.motor
&
fpsensor1
&
softioc-fpsensor
&
XF:05IDD-ES:1\{FPS:1-Chan1\}Pos-I
\\
\hline
testmotor
&
bs.motor
&
fpsensor1
&
softioc-fpsensor
&
XF:05IDD-ES:1\{FPS:1-Chan2\}Pos-I
\\
\hline
nanoKBv Fine Pitch
&
bs.motor
&
PI E518
&
softioc-mcd19
&
XF:05IDD-ES:1\{nKB:vert-Ax:PFPI\}Mtr
\\
\hline
nanoKBh Fine Pitch
&
bs.motor
&
PI E518
&
softioc-mcd19
&
XF:05IDD-ES:1\{nKB:horz-Ax:PFPI\}Mtr
\\
\hline
nanoKBh Coarse Pitch
&
bs.motor
&
PI E712
&
softioc-mcd20
&
XF:05IDD-ES:1\{nKB:horz-Ax:PC\}Mtr
\\
\hline
nanoKBv Coarse Pitch
&
bs.motor
&
PI E712
&
softioc-mcd24
&
XF:05IDD-ES:1\{nKB:vert-Ax:PC\}Mtr
\\
\hline
Sample Coarse Z
&
nano\_stage.z
&
Smaract
&
softioc-mcd26
&
XF:05IDD-ES:1\{nKB:Smpl-Ax:sz\}Mtr
\\
\hline
Sample Coarse X
&
nano\_stage.x
&
Smaract
&
softioc-mcd26
&
XF:05IDD-ES:1\{nKB:Smpl-Ax:sx\}Mtr
\\
\hline
Sample Coarse Y
&
nano\_stage.y
&
Smaract
&
softioc-mcd26
&
XF:05IDD-ES:1\{nKB:Smpl-Ax:sy\}Mtr
\\
\hline
Sample Theta
&
nano\_stage.th
&
Smaract
&
softioc-mcd26
&
XF:05IDD-ES:1\{nKB:Smpl-Ax:th\}Mtr
\\
\hline
Sample Top Z
&
nano\_stage.topx
&
Smaract
&
softioc-mcd26
&
XF:05IDD-ES:1\{nKB:Smpl-Ax:zth\}Mtr
\\
\hline
Sample Top X
&
nano\_stage.topz
&
Smaract
&
softioc-mcd26
&
XF:05IDD-ES:1\{nKB:Smpl-Ax:xth\}Mtr
\\
\hline
Sample Scanner X
&
nano\_stage.sx
&
nPoint
&
softioc-nPoint
&
XF:05IDD-ES:1\{nKB:Smpl-Ax:ssx\}Mtr
\\
\hline
Sample Scanner Y
&
nano\_stage.sy
&
nPoint
&
softioc-nPoint
&
XF:05IDD-ES:1\{nKB:Smpl-Ax:ssy\}Mtr
\\
\hline
Sample Scanner Z
&
nano\_stage.sz
&
nPoint
&
softioc-nPoint
&
XF:05IDD-ES:1\{nKB:Smpl-Ax:ssz\}Mtr
\\
\hline
testmotor
&
bs.motor
&
picoscale
&
softioc-picoscale
&
XF:
\\
\hline
nanoZebra
&
nanoZebra
&
none
&
softioc-zebra
&
XF:05IDD-ES:1\{Dev:Zebra2\}
\\
\hline
\end{tabulary}
\par
\sphinxattableend\end{savenotes}


\begin{savenotes}\sphinxattablestart
\raggedright
\sphinxcapstartof{table}
\sphinxthecaptionisattop
\sphinxcaption{xf05idd-det1}\label{\detokenize{staff:xf05idd-det1}}
\sphinxaftertopcaption
\begin{tabulary}{\linewidth}[t]{|T|T|T|T|T|}
\hline
\sphinxstyletheadfamily 
Motor Controller
&\sphinxstyletheadfamily 
IOC
&\sphinxstyletheadfamily 
Motor
&\sphinxstyletheadfamily 
PV
&\sphinxstyletheadfamily 
Bluesky Object
\\
\hline
mc01
&
softioc-mc01
&
testmotor
&
XF:
&
bs.motor
\\
\hline
\end{tabulary}
\par
\sphinxattableend\end{savenotes}


\begin{savenotes}\sphinxattablestart
\raggedright
\sphinxcapstartof{table}
\sphinxthecaptionisattop
\sphinxcaption{xf05idd-det2}\label{\detokenize{staff:xf05idd-det2}}
\sphinxaftertopcaption
\begin{tabulary}{\linewidth}[t]{|T|T|T|T|T|}
\hline
\sphinxstyletheadfamily 
Motor Controller
&\sphinxstyletheadfamily 
IOC
&\sphinxstyletheadfamily 
Motor
&\sphinxstyletheadfamily 
PV
&\sphinxstyletheadfamily 
Bluesky Object
\\
\hline
mc01
&
softioc-mc01
&
testmotor
&
XF:
&
bs.motor
\\
\hline
\end{tabulary}
\par
\sphinxattableend\end{savenotes}


\begin{savenotes}\sphinxattablestart
\raggedright
\sphinxcapstartof{table}
\sphinxthecaptionisattop
\sphinxcaption{xf05idd-ioc2}\label{\detokenize{staff:xf05idd-ioc2}}
\sphinxaftertopcaption
\begin{tabulary}{\linewidth}[t]{|T|T|T|T|T|}
\hline
\sphinxstyletheadfamily 
Motor Controller
&\sphinxstyletheadfamily 
IOC
&\sphinxstyletheadfamily 
Motor
&\sphinxstyletheadfamily 
PV
&\sphinxstyletheadfamily 
Bluesky Object
\\
\hline
mc01
&
softioc-mc01
&
testmotor
&
XF:
&
bs.motor
\\
\hline
\end{tabulary}
\par
\sphinxattableend\end{savenotes}


\subsection{EPS}
\label{\detokenize{staff:eps}}
\begin{sphinxadmonition}{note}{\label{staff:index-5}Todo:}\begin{itemize}
\item {} 
Upload wiring diagrams

\end{itemize}
\end{sphinxadmonition}


\chapter{Indices and tables}
\label{\detokenize{index:indices-and-tables}}\begin{itemize}
\item {} 
\DUrole{xref,std,std-ref}{genindex}

\item {} 
\DUrole{xref,std,std-ref}{modindex}

\item {} 
\DUrole{xref,std,std-ref}{search}

\end{itemize}



\renewcommand{\indexname}{Index}
\printindex
\end{document}